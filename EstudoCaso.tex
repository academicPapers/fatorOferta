\PassOptionsToPackage{unicode=true}{hyperref} % options for packages loaded elsewhere
\PassOptionsToPackage{hyphens}{url}
\documentclass[9pt,ignorenonframetext,aspectratio=169]{beamer}
\IfFileExists{pgfpages.sty}{\usepackage{pgfpages}}{}
\setbeamertemplate{caption}[numbered]
\setbeamertemplate{caption label separator}{: }
\setbeamercolor{caption name}{fg=normal text.fg}
\beamertemplatenavigationsymbolsempty
\usepackage{lmodern}
\usepackage{amssymb,amsmath}
\usepackage{ifxetex,ifluatex}
\usepackage{fixltx2e} % provides \textsubscript
\ifnum 0\ifxetex 1\fi\ifluatex 1\fi=0 % if pdftex
  \usepackage[T1]{fontenc}
  \usepackage[utf8]{inputenc}
\else % if luatex or xelatex
  \ifxetex
    \usepackage{mathspec}
  \else
    \usepackage{fontspec}
\fi
\defaultfontfeatures{Ligatures=TeX,Scale=MatchLowercase}







\fi

  \usetheme[]{metropolis}






% use upquote if available, for straight quotes in verbatim environments
\IfFileExists{upquote.sty}{\usepackage{upquote}}{}
% use microtype if available
\IfFileExists{microtype.sty}{%
  \usepackage{microtype}
  \UseMicrotypeSet[protrusion]{basicmath} % disable protrusion for tt fonts
}{}


\newif\ifbibliography
  \usepackage[style=abnt,]{biblatex}
      \addbibresource{references.bib}
  

\hypersetup{
      pdftitle={Implicações da aplicação do fator oferta},
        pdfauthor={Luiz Fernando Palin Droubi ; Carlos Augusto Zilli ; Willian Zonato ; Norberto Hochheim},
          pdfborder={0 0 0},
    breaklinks=true}
%\urlstyle{same}  % Use monospace font for urls




  \usepackage{color}
  \usepackage{fancyvrb}
  \newcommand{\VerbBar}{|}
  \newcommand{\VERB}{\Verb[commandchars=\\\{\}]}
  \DefineVerbatimEnvironment{Highlighting}{Verbatim}{commandchars=\\\{\}}
  % Add ',fontsize=\small' for more characters per line
  \usepackage{framed}
  \definecolor{shadecolor}{RGB}{248,248,248}
  \newenvironment{Shaded}{\begin{snugshade}}{\end{snugshade}}
  \newcommand{\AlertTok}[1]{\textcolor[rgb]{0.94,0.16,0.16}{#1}}
  \newcommand{\AnnotationTok}[1]{\textcolor[rgb]{0.56,0.35,0.01}{\textbf{\textit{#1}}}}
  \newcommand{\AttributeTok}[1]{\textcolor[rgb]{0.77,0.63,0.00}{#1}}
  \newcommand{\BaseNTok}[1]{\textcolor[rgb]{0.00,0.00,0.81}{#1}}
  \newcommand{\BuiltInTok}[1]{#1}
  \newcommand{\CharTok}[1]{\textcolor[rgb]{0.31,0.60,0.02}{#1}}
  \newcommand{\CommentTok}[1]{\textcolor[rgb]{0.56,0.35,0.01}{\textit{#1}}}
  \newcommand{\CommentVarTok}[1]{\textcolor[rgb]{0.56,0.35,0.01}{\textbf{\textit{#1}}}}
  \newcommand{\ConstantTok}[1]{\textcolor[rgb]{0.00,0.00,0.00}{#1}}
  \newcommand{\ControlFlowTok}[1]{\textcolor[rgb]{0.13,0.29,0.53}{\textbf{#1}}}
  \newcommand{\DataTypeTok}[1]{\textcolor[rgb]{0.13,0.29,0.53}{#1}}
  \newcommand{\DecValTok}[1]{\textcolor[rgb]{0.00,0.00,0.81}{#1}}
  \newcommand{\DocumentationTok}[1]{\textcolor[rgb]{0.56,0.35,0.01}{\textbf{\textit{#1}}}}
  \newcommand{\ErrorTok}[1]{\textcolor[rgb]{0.64,0.00,0.00}{\textbf{#1}}}
  \newcommand{\ExtensionTok}[1]{#1}
  \newcommand{\FloatTok}[1]{\textcolor[rgb]{0.00,0.00,0.81}{#1}}
  \newcommand{\FunctionTok}[1]{\textcolor[rgb]{0.00,0.00,0.00}{#1}}
  \newcommand{\ImportTok}[1]{#1}
  \newcommand{\InformationTok}[1]{\textcolor[rgb]{0.56,0.35,0.01}{\textbf{\textit{#1}}}}
  \newcommand{\KeywordTok}[1]{\textcolor[rgb]{0.13,0.29,0.53}{\textbf{#1}}}
  \newcommand{\NormalTok}[1]{#1}
  \newcommand{\OperatorTok}[1]{\textcolor[rgb]{0.81,0.36,0.00}{\textbf{#1}}}
  \newcommand{\OtherTok}[1]{\textcolor[rgb]{0.56,0.35,0.01}{#1}}
  \newcommand{\PreprocessorTok}[1]{\textcolor[rgb]{0.56,0.35,0.01}{\textit{#1}}}
  \newcommand{\RegionMarkerTok}[1]{#1}
  \newcommand{\SpecialCharTok}[1]{\textcolor[rgb]{0.00,0.00,0.00}{#1}}
  \newcommand{\SpecialStringTok}[1]{\textcolor[rgb]{0.31,0.60,0.02}{#1}}
  \newcommand{\StringTok}[1]{\textcolor[rgb]{0.31,0.60,0.02}{#1}}
  \newcommand{\VariableTok}[1]{\textcolor[rgb]{0.00,0.00,0.00}{#1}}
  \newcommand{\VerbatimStringTok}[1]{\textcolor[rgb]{0.31,0.60,0.02}{#1}}
  \newcommand{\WarningTok}[1]{\textcolor[rgb]{0.56,0.35,0.01}{\textbf{\textit{#1}}}}



% Prevent slide breaks in the middle of a paragraph:
\widowpenalties 1 10000
\raggedbottom

  \AtBeginPart{
    \let\insertpartnumber\relax
    \let\partname\relax
    \frame{\partpage}
  }
  \AtBeginSection{
    \ifbibliography
    \else
      \let\insertsectionnumber\relax
      \let\sectionname\relax
      \frame{\sectionpage}
    \fi
  }
  \AtBeginSubsection{
    \let\insertsubsectionnumber\relax
    \let\subsectionname\relax
    \frame{\subsectionpage}
  }



\setlength{\parindent}{0pt}
\setlength{\parskip}{6pt plus 2pt minus 1pt}
\setlength{\emergencystretch}{3em}  % prevent overfull lines
\providecommand{\tightlist}{%
  \setlength{\itemsep}{0pt}\setlength{\parskip}{0pt}}

  \setcounter{secnumdepth}{0}


  \usepackage[brazil]{babel}
  \usepackage{csquotes}

  \title[]{Implicações da aplicação do fator oferta}

  \subtitle{Em modelos estatísticos}

  \author[
        Luiz Fernando Palin Droubi\footnote<.->{\href{mailto:lfpdroubi@gmail.com}{\nolinkurl{lfpdroubi@gmail.com}}}
\newline \and Carlos Augusto Zilli\footnote<.->{\href{mailto:carlos.zilli@ifsc.edu.br}{\nolinkurl{carlos.zilli@ifsc.edu.br}}}
\newline \and Willian Zonato\footnote<.->{\href{mailto:will.zonato@gmail.com}{\nolinkurl{will.zonato@gmail.com}}}
\newline \and Norberto Hochheim\footnote<.->{\href{mailto:norberto.hochheim@ufsc.br}{\nolinkurl{norberto.hochheim@ufsc.br}}}
    ]{Luiz Fernando Palin Droubi\footnote<.->{\href{mailto:lfpdroubi@gmail.com}{\nolinkurl{lfpdroubi@gmail.com}}}
\newline \and Carlos Augusto Zilli\footnote<.->{\href{mailto:carlos.zilli@ifsc.edu.br}{\nolinkurl{carlos.zilli@ifsc.edu.br}}}
\newline \and Willian Zonato\footnote<.->{\href{mailto:will.zonato@gmail.com}{\nolinkurl{will.zonato@gmail.com}}}
\newline \and Norberto Hochheim\footnote<.->{\href{mailto:norberto.hochheim@ufsc.br}{\nolinkurl{norberto.hochheim@ufsc.br}}}}

  \institute[
    ]{
    GEAP - UFSC
    }

\date[
      \today
  ]{
      \today
        }


\begin{document}

% Hide progress bar and footline on titlepage
  \begin{frame}[plain]
  \titlepage
  \end{frame}


  \begin{frame}
  \tableofcontents[hideallsubsections]
  \end{frame}

\hypertarget{estudo-de-caso}{%
\section{Estudo de Caso}\label{estudo-de-caso}}

\begin{frame}{Dados}
\protect\hypertarget{dados}{}

\begin{table}[!htbp] \centering 
  \caption{} 
  \label{} 
\begin{tabular}{@{\extracolsep{5pt}}lccccccc} 
\\[-1.8ex]\hline 
\hline \\[-1.8ex] 
Statistic & valor & area\_total & quartos & suites & garagens & dist\_b\_mar & padrao \\ 
\hline \\[-1.8ex] 
N & 50 & 53 & 53 & 53 & 53 & 53 & 53 \\ 
Mean & 953,800.000 & 188.122 & 2.679 & 1.189 & 1.698 & 528.792 & 2.321 \\ 
St. Dev. & 627,318.800 & 116.215 & 0.754 & 0.900 & 0.972 & 308.098 & 0.754 \\ 
Min & 195,000.000 & 48 & 1 & 0 & 0 & 60 & 1 \\ 
Pctl(25) & 547,750.000 & 109 & 2 & 1 & 1 & 260 & 2 \\ 
Pctl(75) & 1,254,000.000 & 220 & 3 & 1 & 2 & 730 & 3 \\ 
Max & 3,000,000.000 & 578 & 4 & 3 & 4 & 1,430 & 3 \\ 
\hline \\[-1.8ex] 
\end{tabular} 
\end{table}

\end{frame}

\begin{frame}[fragile]{Variável resposta}
\protect\hypertarget{variuxe1vel-resposta}{}

\begin{itemize}
\tightlist
\item
  Valores de oferta
\end{itemize}

\begin{Shaded}
\begin{Highlighting}[]
\KeywordTok{var}\NormalTok{(dados}\OperatorTok{$}\NormalTok{valor, }\DataTypeTok{na.rm =} \OtherTok{TRUE}\NormalTok{)}
\end{Highlighting}
\end{Shaded}

\begin{verbatim}
## [1] 393528897959
\end{verbatim}

\begin{itemize}
\tightlist
\item
  Valores de oferta ajustados
\end{itemize}

\begin{Shaded}
\begin{Highlighting}[]
\KeywordTok{var}\NormalTok{(.}\DecValTok{9}\OperatorTok{*}\NormalTok{dados}\OperatorTok{$}\NormalTok{valor, }\DataTypeTok{na.rm =} \OtherTok{TRUE}\NormalTok{)}
\end{Highlighting}
\end{Shaded}

\begin{verbatim}
## [1] 318758407347
\end{verbatim}

\begin{itemize}
\tightlist
\item
  Ajuste da variância
\end{itemize}

\begin{Shaded}
\begin{Highlighting}[]
\FloatTok{.9}\OperatorTok{*}\NormalTok{.}\DecValTok{9}\OperatorTok{*}\KeywordTok{var}\NormalTok{(dados}\OperatorTok{$}\NormalTok{valor, }\DataTypeTok{na.rm =} \OtherTok{TRUE}\NormalTok{)}
\end{Highlighting}
\end{Shaded}

\begin{verbatim}
## [1] 318758407347
\end{verbatim}

\end{frame}

\begin{frame}[fragile]{Ajuste de modelos}
\protect\hypertarget{ajuste-de-modelos}{}

\begin{itemize}
\tightlist
\item
  Com dados de oferta
\end{itemize}

\begin{Shaded}
\begin{Highlighting}[]
\NormalTok{fit <-}\StringTok{ }\KeywordTok{lm}\NormalTok{(}\KeywordTok{log}\NormalTok{(valor)}\OperatorTok{~}\NormalTok{area_total }\OperatorTok{+}\StringTok{ }\NormalTok{quartos }\OperatorTok{+}\StringTok{ }\NormalTok{suites }\OperatorTok{+}\StringTok{ }\NormalTok{garagens }\OperatorTok{+}\StringTok{ }
\StringTok{            }\KeywordTok{log}\NormalTok{(dist_b_mar) }\OperatorTok{+}\StringTok{ }\KeywordTok{I}\NormalTok{(padrao}\OperatorTok{^-}\DecValTok{1}\NormalTok{), }\DataTypeTok{data =}\NormalTok{ dados, }\DataTypeTok{subset =} \OperatorTok{-}\KeywordTok{c}\NormalTok{(}\DecValTok{31}\NormalTok{, }\DecValTok{39}\NormalTok{))}
\end{Highlighting}
\end{Shaded}

\begin{itemize}
\tightlist
\item
  Com dados de oferta pré-ajustados
\end{itemize}

\begin{Shaded}
\begin{Highlighting}[]
\NormalTok{dados <-}\StringTok{ }\KeywordTok{within}\NormalTok{(dados, valor1 <-}\StringTok{ }\FloatTok{.9}\OperatorTok{*}\NormalTok{valor)}
\NormalTok{fit1 <-}\StringTok{ }\KeywordTok{lm}\NormalTok{(}\KeywordTok{log}\NormalTok{(valor1)}\OperatorTok{~}\NormalTok{area_total }\OperatorTok{+}\StringTok{ }\NormalTok{quartos }\OperatorTok{+}\StringTok{ }\NormalTok{suites }\OperatorTok{+}\StringTok{ }\NormalTok{garagens }\OperatorTok{+}\StringTok{ }
\StringTok{            }\KeywordTok{log}\NormalTok{(dist_b_mar) }\OperatorTok{+}\StringTok{ }\KeywordTok{I}\NormalTok{(padrao}\OperatorTok{^-}\DecValTok{1}\NormalTok{), }\DataTypeTok{data =}\NormalTok{ dados, }\DataTypeTok{subset =} \OperatorTok{-}\KeywordTok{c}\NormalTok{(}\DecValTok{31}\NormalTok{, }\DecValTok{39}\NormalTok{))}
\end{Highlighting}
\end{Shaded}

\end{frame}

\begin{frame}[fragile]{Previsões com os modelos}
\protect\hypertarget{previsuxf5es-com-os-modelos}{}

\begin{itemize}
\tightlist
\item
  Com dados de oferta
\end{itemize}

\begin{Shaded}
\begin{Highlighting}[]
\NormalTok{new <-}\StringTok{ }\NormalTok{dados[}\DecValTok{52}\NormalTok{, ]}
\NormalTok{p <-}\StringTok{ }\KeywordTok{predict}\NormalTok{(fit, }\DataTypeTok{newdata =}\NormalTok{ new, }\DataTypeTok{interval =} \StringTok{"confidence"}\NormalTok{, }\DataTypeTok{level =} \FloatTok{.80}\NormalTok{)}
\NormalTok{(P <-}\StringTok{ }\FloatTok{.9}\OperatorTok{*}\KeywordTok{exp}\NormalTok{(p))}
\end{Highlighting}
\end{Shaded}

\begin{verbatim}
##         fit      lwr      upr
## 52 865494.6 832291.3 900022.4
\end{verbatim}

\begin{itemize}
\tightlist
\item
  Com dados de oferta pré-ajustados
\end{itemize}

\begin{Shaded}
\begin{Highlighting}[]
\NormalTok{p1 <-}\StringTok{ }\KeywordTok{predict}\NormalTok{(fit1, }\DataTypeTok{newdata =}\NormalTok{ new, }\DataTypeTok{interval =} \StringTok{"confidence"}\NormalTok{, }\DataTypeTok{level =} \FloatTok{.80}\NormalTok{)}
\NormalTok{(P1 <-}\StringTok{ }\KeywordTok{exp}\NormalTok{(p1))}
\end{Highlighting}
\end{Shaded}

\begin{verbatim}
##         fit      lwr      upr
## 52 865494.6 832291.3 900022.4
\end{verbatim}

\end{frame}

\hypertarget{conclusuxe3o}{%
\section{Conclusão}\label{conclusuxe3o}}

\begin{frame}{Conclusão}

\begin{itemize}
\tightlist
\item
  Quando da utilização de dados de oferta para elaboração de modelos,
  além do valor da estimativa central, também os limites do intervalo de
  confiança devem ser ajustados pelo fator oferta.
\end{itemize}

\end{frame}


  \begin{frame}[allowframebreaks]{}
  \bibliographytrue
  \printbibliography[heading=none]
  \end{frame}


\end{document}
